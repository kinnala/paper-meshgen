%\documentclass[12pt]{rmstyle}
\documentclass[12pt]{article}
%\usepackage[utf8]{inputenc}
\usepackage[T1]{fontenc}
\usepackage{graphicx}
\usepackage{grffile}
\usepackage{longtable}
\usepackage{wrapfig}
\usepackage{rotating}
\usepackage[normalem]{ulem}
\usepackage{amsmath}
\usepackage{textcomp}
\usepackage{amssymb}
\usepackage{capt-of}
\usepackage{hyperref}
\usepackage{stmaryrd}
\usepackage{algorithm}
\usepackage{xcolor}
\usepackage{algorithmicx}
\usepackage[noend]{algpseudocode}
\author{Tom Gustafsson\footnote{Corresponding author. \texttt{firstname.lastname@aalto.fi}}}
\date{\today}
 \hypersetup{
     colorlinks=true,
     linkcolor=purple,
     filecolor=purple,
     citecolor=purple,
     urlcolor=purple,
     }
 \title{A simple technique for unstructured mesh generation via adaptive finite elements}


 \makeatletter%
\@ifclassloaded{<rmstyle>}%
  {\volume{XX}
\issue{Y}
\year{20ZZ}
\doilabel{12345}}%
  {}%
\makeatother%
 
\begin{document}

\newcommand*\DNA{\textsc{dna}}

\newcommand*\Let[2]{\State #1 $\gets$ #2}
\algrenewcommand\algorithmicrequire{\textbf{Precondition:}}
\algrenewcommand\algorithmicensure{\textbf{Postcondition:}}

\maketitle

\begin{abstract}
  This article describes a concise algorithm for the generation of triangular
  meshes with the help of standard adaptive finite element methods.  We
  demonstrate that a generic adaptive finite element solver can be repurposed
  into a triangular mesh generator if a robust mesh smoothing algorithm is
  applied between the mesh refinement steps.  We present an implementation of
  the mesh generator and apply it in the creation of a mesh for the finite
  volume simulation of an experimental creeping flow measurement analogous to
  electron transport in graphene.
\end{abstract}

\section{Introduction}
\label{sec:orge4667b0}

Many numerical methods for partial differential equations (PDE's), such as the
finite element method (FEM) and the finite volume method (FVM), are based on
splitting the domain of the solution into primitive shapes.  The collection of
the primitive shapes, i.e.~the computational mesh, is used to define the
discretization, e.g., in the FEM, the shape functions are polynomial in each
element of the mesh, and in the FVM, the discrete fluxes are defined over the
edges of the mesh.

This article describes a simple approach for the triangulation of
two-dimensional polygonal domains that are suitable for the discretization of
PDE's.  The process can be summarized as follows:
\begin{enumerate}
\item Find a constrained Delaunay triangulation (CDT) of the polygonal domain
      using the corner points as input vertices and the domain boundaries
      as constraints.
\item Solve the Laplace equation with the given triangulation
      and the FEM.
\item Split
      the triangles with the largest error indicator
      using adaptive mesh refinement techniques.
\item Apply centroidal patch tesselation (CPT) smoothing to the resulting
  triangulation.
\item Go to step 2.
\end{enumerate}
It is noteworthy that the steps 2, 3 and 5 correspond exactly to what is done in
any standard implementation of the adaptive FEM;
cf.~Verf\"{u}rth~\cite{Verf_rth_2013} who calls it the \emph{adaptive process}.

The goal of this work is to demonstrate that if the mesh smoothing algorithm of
step 4 is chosen properly, the adaptive process tends to produce reasonable
meshes even if the initial mesh is of low quality.  Thus, we demonstrate that
the adaptive process---together with an implementation of the CDT and a robust
mesh smoothing algorithm---can act as a simple triangular mesh generator.

\section{Prior work}
\label{sec:org7798f6d}

Two popular techniques for generating unstructured meshes are those based on the
\emph{advancing front technique}~\cite{L_hner_1988} or the \emph{Delaunay mesh
refinement}~\cite{Chew_1989, Ruppert_1995, Shewchuk_2002}.  In addition, there
exists several less known techniques such as \emph{quadtree
meshing}~\cite{Yerry_1983}, \emph{bubble packing}~\cite{Shimada_1995}, and
hybrid techniques combining some of the above~\cite{mavriplis1995advancing}.

Some existing techniques bear similarity to ours.  For example,
Bossen--Heckbert~\cite{bossen1996pliant} start also with a CDT and improve it by
relocating the nodes.  Instead of randomly picking nodes for relocation, we use
the finite element error indicator that guides the refinement.  Instead of doing
local modifications, we split simultaneously all triangles that have their error indicators
above a predefined threshold.

Persson--Strang~\cite{persson2004simple} describe another technique based on
iterative relocation of the nodes.  An initial mesh is given by a structured
background mesh which is then relaxed by intepreting the edges as a
precompressed truss structure.  The structure is then forced inside a given
computational domain by expressing the boundary using signed distance functions
and interpreting the signed distance as an external load acting on the truss.
In contrast to our approach, the geometry description is implicit, i.e.~the
boundary is defined as the zero set of a user-defined function.

\section{Components of the mesh generator}
\label{sec:components}

The input to
our mesh generator is a sequence of \(N\) corner points
$$\mathcal{C} = (\mathcal{C}_1, \mathcal{C}_2, \dots, \mathcal{C}_N), \quad \mathcal{C}_j \in \mathbb{R}^2, \quad j = 1,\dots,N,$$
that form a polygon when connected by the edges
$$(\mathcal{C}_1, \mathcal{C}_2), (\mathcal{C}_2,\mathcal{C}_3),
\dots, (\mathcal{C}_{N-1}, \mathcal{C}_N), (\mathcal{C}_N,\mathcal{C}_1).$$
We do not allow self-intersecting polygons although
the algorithm can be generalized to polygons
with holes.
The corresponding polygonal domain is denoted
by \(\Omega_{\mathcal{C}} \subset \mathbb{R}^2\).

\subsection{Constrained Delaunay triangulation}
\label{sec:cdt}

A \emph{triangulation} of \(\Omega_{\mathcal{C}}\) is a
collection of nonoverlapping nondegenerate triangles whose union is exactly
\(\Omega_{\mathcal{C}}\).  Our initial triangulation \(\mathcal{T}_0\) is a
\emph{constrained Delaunay triangulation} (CDT) of the input vertices
\(\mathcal{C}\) with the edges \((\mathcal{C}_1, \mathcal{C}_2)\),
\((\mathcal{C}_2,\mathcal{C}_3)\), \(\dots\), \((\mathcal{C}_{N-1},
\mathcal{C}_N)\), \((\mathcal{C}_N, \mathcal{C}_1)\) constrained to be a part of
the resulting triangulation and the triangles outside the polygon removed;
cf.~Chew~\cite{Chew_1987} for the exact definition of a CDT and an algorithm for
its construction.

An example initial triangulation of a polygon with a spiral-shaped boundary is
given in Figure~\ref{fig:cdt}. It is obvious that the CDT is not always a high
quality computational mesh due to the presence of arbitrarily small angles.
Thus, we seek to improve the initial triangulation by iteratively adding new
triangles, and smoothing the mesh.  Note that the remaining steps do not assume
the use of CDT as an initial triangulation---any triangulation with the
prescribed edges will suffice.

\begin{figure}[htbp]
  \centering
  \includegraphics[width=0.24\textwidth]{../image_19.png}
  \includegraphics[width=0.24\textwidth]{../image_5.png}
\caption{A spiral-shaped boundary approximated by linear segments and the
  corresponding CDT with the triangles outside of the polygon removed.  An
  example from the documentation of the Triangle mesh
  generator~\cite{shewchuk1996triangle}.}
\label{fig:cdt}
\end{figure}

\subsection{Solving the Laplace equation}
\label{sec:poisson}

In order to decide on the placement of the new vertices and triangles, we solve
the Laplace equation\footnote{The choice of the Laplace equation is motivated by
the following heuristic observation: a \emph{quality mesh} is often synonymous
with a \emph{good mesh for the finite element solution of the Laplace
equation}.} using the FEM and evalute the corresponding a~posteriori error
estimator.  The triangles that have the highest values of the error estimator
are refined, i.e.~split into smaller triangles.

The Laplace equation reads: find $u : \Omega_{\mathcal{C}} \rightarrow \mathbb{R}$ satisfying
\begin{alignat}{2}
-\Delta u &= 1 \quad && \text{in $\Omega_{\mathcal{C}}$,} \\
u &= 0 \quad && \text{on $\partial \Omega_{\mathcal{C}}$.}
\end{alignat}
The finite element method is used to numerically solve the
weak formulation: find \(u \in V\) such that
\begin{equation}
   \label{eq:weakform}
   \int_{\Omega_{\mathcal{C}}} \nabla u \cdot \nabla v \,\mathrm{d}x = \int_{\Omega_{\mathcal{C}}} v\,\mathrm{d}x \quad \forall v \in V,
\end{equation}
where
\(w \in V\) if \(w |_{\partial \Omega_{\mathcal{C}}} = 0\) and
$
   \int_{\Omega_{\mathcal{C}}} (\nabla w)^2 \,\mathrm{d}x < \infty.
$

We denote the \(k\)th triangulation of the
domain \(\Omega_{\mathcal{C}}\) by \(\mathcal{T}_k\), \(k=0,1,\dots\), and
use the piecewise linear polynomial space
$$V_h^k = \{ v \in V : v|_T \in P_1(T)~\forall T \in \mathcal{T}_k \},$$
where $P_1(T)$ denotes the set of linear polynomials over $T$.
The finite element method corresponding to the \(k\)th iteration reads:
find \(u_h^k \in V_h^k\) such that
\begin{equation}
   \label{eq:discweakform}
   \int_{\Omega_{\mathcal{C}}} \nabla u_h^k \cdot \nabla v_h \,\mathrm{d}x = \int_{\Omega_{\mathcal{C}}} v_h\,\mathrm{d}x \quad \forall v_h \in V_h^k.
\end{equation}
The local a posteriori error estimator
reads
\begin{equation}
        \eta_T(u_h^k) = \sqrt{h_T^2 A_T^2 + \frac12 h_T \int_{\partial T \setminus \partial \Omega_{\mathcal{C}}} (\llbracket \nabla u_h^k \cdot \boldsymbol{n} \rrbracket)^2 \,\mathrm{d}s}, \quad T \in \mathcal{T}_k,
\end{equation}
where $A_T$ is the area of the triangle $T$ and $h_T$ is the length of its longest edge, $\llbracket w \rrbracket |_{\partial T \setminus \partial \Omega_{\mathcal{C}}}$ denotes the jump in the values of
$w$ over $\partial T \setminus \partial \Omega_{\mathcal{C}}$, and $\boldsymbol{n}$ is a unit normal vector to
$\partial T$.
The error estimator $\eta_T$ is evaluated for each triangle
after solving \eqref{eq:discweakform}.
Finally, a triangle $T \in \mathcal{T}_k$ is marked for refinement if
\begin{equation}
  \label{eq:adaptivetheta}
   \eta_T > \theta \max_{T^\prime \in \mathcal{T}_k} \eta_{T^\prime},
\end{equation}
where $0 < \theta < 1$ is a parameter controlling the amount
of elements to split during each iteration.

\subsection{Red-green-blue refinement}
\label{sec:rgb}

The triangles marked for refinement by the rule \eqref{eq:adaptivetheta} are
split into four.  In order to keep the rest of the triangulation conformal,
i.e.~to not have any hanging nodes, the neighboring triangles are split into two or
three by the so-called red-green-blue (RGB)
refinement; cf.~Carstensen~\cite{carstensen2004adaptive}.
Using RGB refinement to the example of Figure~\ref{fig:cdt}
is depicted in Figure~\ref{fig:firstrgb}.

\begin{figure}[htbp]
\centering
\includegraphics[width=0.24\textwidth]{../image_5.png}
\includegraphics[width=0.24\textwidth]{../image_6.png}
\caption{(Left.) The initial triangulation. (Right.) The resulting triangulation
  after a solve of \eqref{eq:discweakform} and an adaptive RGB refinement.}
\label{fig:firstrgb}
\end{figure}

\subsection{Centroidal patch triangulation smoothing}
\label{sec:cpt}

We use a mesh smoothing approach introduced by Chen--Holst~\cite{Chen_2011} who
refer to the algorithm as \emph{centroidal patch triangulation} (CPT) smoothing.
The idea is to repeatedly move the interior vertices to the area-weighted
averages of the barycenters of the surrounding triangles.  The CPT smoothing is
combined with an edge flipping algorithm, also described in
Chen--Holst~\cite{Chen_2011}, to improve the quality of the resulting
triangulation.  The mesh smoother is applied to the spiral-shaped domain example 
in Figure~\ref{fig:firstsmooth}.

%% \begin{algorithm}[H]
%%   \caption{Mesh smoother}
%%   \begin{algorithmic}[1]
%%     \Require{$\mathcal{T}$ is a triangular mesh}
%%     \Require{$S$ is the total number of smoothing iterations}
%%     \Statex
%%     \Function{CPT}{$\mathcal{T}$}
%%     \Let{$\mathcal{T}_0$}{$\textsc{CDT}(\mathcal{C}$)}
%%     \For{$k \gets 1 \textrm{ to } M$}
%%     \Let{$\mathcal{T}_k^\prime$}{$\textsc{RGB}(\mathcal{T}_{k-1},\{\eta_T : T \in \mathcal{T}_{k-1}\})$}
%%     \Let{$\mathcal{T}_k$}{$\textsc{CPT}(\mathcal{T}_k^\prime)$}
%%     \EndFor
%%     \State \Return{$\mathcal{T}_N$}
%%     \EndFunction
%%   \end{algorithmic}
%% \end{algorithm}

\begin{figure}[htbp]
\centering
\includegraphics[width=0.24\textwidth]{../image_6.png}
\includegraphics[width=0.24\textwidth]{../image_7.png}
\caption{(Left.) Once adaptively refined triangulation. (Right.) The resulting
  triangulation after smoothing and edge flipping.}
\label{fig:firstsmooth}
\end{figure}

\section{The mesh generation algorithm}
\label{sec:orgff9b6c1}

In previous sections we gave an overview of all the components of the mesh
generation algorithm.  The resulting mesh generator is now summarized in
Algorithm~\ref{alg:meshgen}.  The total number of refinements $M$ is a constant
in order to guarantee the termination of the algorithm.  Nevertheless, in
practice and in our implementation the refinement loop is terminated when a
quality criterion is satisfied, e.g., when the average minimum angle of the
triangles is above a predefined threshold.  The entire mesh generation process
for the spiral-shaped domain example is given in Figure~\ref{fig:spiralexample}.


\begin{figure}[htbp]
  \centering
  \includegraphics[width=0.24\textwidth]{../image_19.png}
  \includegraphics[width=0.24\textwidth]{../image_5.png}
  \includegraphics[width=0.24\textwidth]{../image_6.png}
  \includegraphics[width=0.24\textwidth]{../image_7.png}\\
  \includegraphics[width=0.24\textwidth]{../image_8.png}
  \includegraphics[width=0.24\textwidth]{../image_9.png}
  \includegraphics[width=0.24\textwidth]{../image_10.png}
  \includegraphics[width=0.24\textwidth]{../image_11.png}\\
  \includegraphics[width=0.24\textwidth]{../image_12.png}
  \includegraphics[width=0.24\textwidth]{../image_13.png}
  \includegraphics[width=0.24\textwidth]{../image_14.png}
  \includegraphics[width=0.24\textwidth]{../image_15.png}\\
  \includegraphics[width=0.24\textwidth]{../image_16.png}
  \includegraphics[width=0.24\textwidth]{../image_17.png}
  \caption{The entire mesh generation process for the spiral-shaped domain
    example from left-to-right, top-to-bottom.}
\label{fig:spiralexample}
\end{figure}


\begin{algorithm}[H]
  \caption{Pseudocode for the triangular mesh generator}
  \label{alg:meshgen}
  \begin{algorithmic}[1]
    \Require{$\mathcal{C}$ is a sequence of polygonal domain corner points}
    \Require{$M$ is the total number of refinements}
    \Statex
    \Function{Generate}{$\mathcal{C}$}
    \Let{$\mathcal{T}_0$}{$\textsc{CDT}(\mathcal{C}$)}
    \For{$k \gets 1 \textrm{ to } M$}
    \Let{$\mathcal{T}_k^\prime$}{$\textsc{RGB}(\mathcal{T}_{k-1},\{\eta_T : T \in \mathcal{T}_{k-1}\})$}
    \Let{$\mathcal{T}_k$}{$\textsc{CPT}(\mathcal{T}_k^\prime)$}
    \EndFor
    \State \Return{$\mathcal{T}_N$}
    \EndFunction
  \end{algorithmic}
\end{algorithm}


\section{Implementation and examples}

We created a prototype implementation of the mesh generator in Python for
computational experiments~\cite{adaptmesh2020}.  The implementation relies
heavily on the scientific Python ecosystem~\cite{virtanen2020scipy}.  It
includes source code from pre-existing Python packages \verb|tri| \cite{tri}
(CDT implementation, ported from Python 2) and the older MIT-licensed versions of
\verb|optimesh| \cite{optimesh} (CPT smoothing) and \verb|meshplex|
\cite{meshplex} (edge flipping).  Moreover, it dynamically imports modules from
\verb|scikit-fem| \cite{gustafsson2020scikit} (RGB refinement) and
\verb|matplotlib| \cite{hunter2007matplotlib} (visualization).

Some example meshes are given in Figure~\ref{fig:moreexamples}.  Note that by
default our implementation uses the average triangle quality\footnote{Triangle
quality is defined as two times the ratio of the incircle and circumcircle
radii.} 0.9 as a stopping criterion which can lead to individual slit triangles.
This is visible especially in the last two examples with randomized boundaries.
This could be circumvented by using a different stopping criterion, e.g., the
minimum triangle angle.  This is not enabled by default because it may increase the
number of triangles significantly.

\begin{figure}[htbp]
  \centering
  \includegraphics[width=0.32\textwidth]{../image_1.png}
  \includegraphics[width=0.32\textwidth]{../image_2.png}
  \includegraphics[width=0.32\textwidth]{../image_4.png}\\
  \includegraphics[width=0.22\textwidth]{../image_21.png}
  \includegraphics[width=0.32\textwidth]{../image_22.png}
  \includegraphics[width=0.32\textwidth]{../image_23.png}
  \caption{Some example meshes generated using \texttt{adaptmesh}---our
    implementation of the proposed algorithm.}
\label{fig:moreexamples}
\end{figure}

\section{An example application}

We simulate an experiment performed on a microfluidic device in
Mayzel--Steinberg--Varshney~\cite{mayzel2019stokes} which is analogous to
electron transport in two-dimensional conductive materials such as graphene.
The governing equation is the Stokes system
\begin{equation}
  \left\{
  \begin{aligned}
    -\nabla \cdot (2\mu\,\varepsilon(\nabla u)) + \nabla p &= 0, \\
    \nabla \cdot u &= 0,
  \end{aligned}
  \right.
\end{equation}
where the unknowns are the velocity $u : \Omega \rightarrow \mathbb{R}^2$ and
the pressure $p : \Omega \rightarrow \mathbb{R}$, the viscosity $\mu > 0$ is
given, and $\varepsilon(\nabla u) = \tfrac12(\nabla u + \nabla u^T)$ is the
rate-of-strain tensor.

The boundary $\partial \Omega$ of the computational domain $\Omega \subset
\mathbb{R}^2$ is extracted from a photograph of the experiment and given as an
input to the mesh generator; the photograph is given in the top-left panel and
the resulting computational mesh in the top-right panel of
Figure~\ref{fig:esimerkki}.  In order to simplify the problem setup we choose
the computational domain so that the left boundary corresponds to the vertical
streamlines in the micro-particle image velocimetry (bottom-left picture),
i.e.~the computational domain is enclosed in the red rectangle.  This allows
using the Dirichlet boundary condition $u_x = 0$, $u_y = -U$, $U > 0$, on the
left boundary.  Our aim is to reproduce the location of the swirl and,
therefore, the exact values $U=\mu=1$ are irrelevant because they only scale the
maximum values of the velocity and the pressure fields in Dirichlet-only
problems, i.e.~the shape of the solution will not change.

\begin{figure}[htbp]
\centering
\includegraphics[width=\textwidth]{./esimerkki_vertailu.png}
\caption{(Top-left) A photograph of the experiment with the particles trapped
  inside a microfluidic device;
  cf.~Mayzel--Steinberg--Varshney~\cite{mayzel2019stokes} for more details.
  (Top-right) The mesh resulting from our mesh generator. (Bottom-left) The
  experimental streamlines obtained from micro-particle image velocimetry.
  (Bottom-right) The numerical velocity field calculated using
  \texttt{FiPy}~\cite{FiPy2009} and the corresponding streamlines from
  \texttt{matplotlib}~\cite{hunter2007matplotlib}.  The images on the left panel
  are cropped from the figures in
  Mayzel--Steinberg--Varshney~\cite{mayzel2019stokes} with the red rectangles
  added later by the author of the present paper.  The original images are
  licensed under Creative Commons Attribution 4.0 International License
  (see~\url{https://creativecommons.org/licenses/by/4.0/}).}
\label{fig:esimerkki}
\end{figure}

\section{Conclusions and future work}
\label{sec:org615e973}

We introduced an algorithm for the generation of triangular meshes for explicit
polygonal domains based on the standard adaptive finite element method and
centroidal patch triangulation smoothing.  We presented an implementation of the
algorithm which was used to demonstrate that the resulting triangular meshes are
reasonable and have an average triangle quality equal to or above 0.9.
The majority of the required components are likely to exist in an
implementation of the adaptive finite element method.  Therefore, the algorithm
may be a suitable candidate if an existing adaptive finite element solver is to be
extended with basic mesh generation capabilities.

\bibliographystyle{unsrt}
\bibliography{mesh}

\end{document}
